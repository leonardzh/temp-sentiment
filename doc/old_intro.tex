% This is the intro section before applying major changes and cuts on 2021-05-04



% Keywords: Text Data, Environmental Shocks, Social Media Mining, Sentiment Analysis, Topic Modeling, Social Media, Public Opinion, Perception, Sentiment, Media, News, Twitter, Climate Change, Science Communication, Policy, Planning, Methods, Text Mining, Natural Language Processing, NLP, Latent Dirichlet Allocation, LDA, Political Ecology, Politics, Resilience, Adaptation, Suicide, Mortality, Morbidity, Self-harm, 

    
%Cite this new paper: https://doi.org/10.1016/S2542-5196(20)30251-5
% And: https://eos.org/articles/long-term-drought-harms-mental-health-in-rural-communities and this is the conf. paper citation https://agu.confex.com/agu/fm20/meetingapp.cgi/Paper/771750
\section{Introduction}

% Organize as follows:

% Heatwaves are known to worsen mental health
% AND mental health is strongly associated with income
% AND wealthier better able to cope with heatwaves 
% BUT we know little about the interaction between heatwaves, mental health and income
% THEREFORE: we did this really novel study.

% Jeremiah Osborne-Gowey: Randy Olsen (of "Don't be such a scientist" fame) and his ABT - And But Therefor framework for communicating - https://www.huffpost.com/entry/and-but-therefore-randy-o_b_8813330
% Jeremiah Osborne-Gowey: https://blogs.scientificamerican.com/observations/how-the-word-but-could-save-the-world/


This work builds on existing work by \citep{baylis_weather_2018}. 




%Intro Paragraph
% I think this should be more about climate change and mental health broadly, as well as a bit about how income has been theorized to affect vulnerability, but there is no evidence yet.  This can be shorter as well, as we have a lot of detail in the Background section 
% Cite this high-level study: https://www.nature.com/articles/s41558-018-0102-4/
% Also this review article saying more research is needed https://www.sciencedirect.com/science/article/pii/S0033350618302130
% -Matt

Research has shown strong linkages between ambient environmental conditions and mental health, with extreme temperatures  frequently associated with poorer overall mental health. Individual's mental health is strongly associated with income with people in lower income levels frequently experiencing higher stress levels and less able to find help for coping with these stresses. Other research on how people cope with environmental temperature extremes indicate that wealthier are better able to cope with heatwaves, whether from traveling to cooler climes, accessing air conditioning, or other temperature ameliorating strategies that may require access to financial capital. Yet we know relatively little about the interaction between temperature, mental health and income. Furthermore, previous studies that did attempt to examine relationships between environmental extremes and health sometimes aggregated data at scales too coarse to examine whether income may be an ameliorating factor. This study examines the interactions between these three using relatively fine-scale, publicly available data. Our research expands on existing studies of temperature and sentiment - one measure of mental health - by examining finer-scale data and incorporating income data to examine correlations between temperature extremes and sentiment.

Human moods and mental states are important aspects of overall well-being and can be influenced by ambient, persistent and fluctuating environmental conditions. Climate change is accelerating the rate and variability around meteorological norms like the timing, magnitude, intensity and duration of precipitation events and air temperature minima and maxima \href{https://www.ipcc.ch/site/assets/uploads/2018/03/SREX-Chap3_FINAL-1.pdf}{(link)}. Previous work \citep{baylis_weather_2018} indicate important linkages between human mood and meteorological conditions. Previous analyses of the effects of weather on mood (as expressed in sentiment of text-based message), however, were constructed at the aggregated city/day level and did not account for differences in socio-economic status which may affect access to resources (e.g., air conditioning) that can offset the effects of exposure to temperature extremes. Here, we build on these previous studies by examining the hourly effects of weather on human sentiment, an expression of mood, exploring geographic and economic heterogeneity at a finer-scale resolution than previous studies.  

% https://medinform.jmir.org/2020/1/e16023/?utm_source=TrendMD&utm_medium=cpc&utm_campaign=JMIR_TrendMD_0
\section{Background}
\subsection{Temperature and Mental Health}
%Jeremiah Osborne-Gowey: We report on physiological and mental health in text here. Maybe just a heading of "health"?
%I think physical health is beyond the scope of this paper -Matt

%https://www.pnas.org/content/115/43/10953
Higher temperatures are associated with many indicators of worsened mental health.  Multiple studies have found strong evidence that higher temperatures are associated with increases in suicides in the United States \citep{Burke2018Aug, Mullins2019Dec, Dixon2007May}, and many others have demonstrated the same relationship around the world \citep{Qi2014Dec, Page2007Aug, Likhvar2011Jan}.  In addition to suicide, other research has found effects temperature on hospitalization events related to mental health disorders such as bipolar disorder and schizophrenia \citep{Mullins2019Dec, Lee2007Jan, Shapira2004Feb, Sung2013Feb, Gupta1992Jun, Hansen2008Oct}, as well as reports of general mental health difficulties \cite{Obradovich2018Oct}.

Temperature has been hypothesized to impact mental health through a number of pathways.  Work on biological mechanisms emphasize that there may be mental health effects from maintaining stable body temperature in high heat \cite{Lohmus2018Jul}.  Additionally, recent studies have found that nighttime temperatures affects sleep quality \citep{Obradovich2017May, Mullins2019Dec}

% The pathways
Higher temperature associated with climate change could influence mental health directly by exposing people to trauma. It could also indirectly influence mental health by affecting (1) physical health and (2)community well-being \citep{RN1314}. For the direct influence pathway, extreme heat events or increasing temperatures have been associated with the increase in aggression,higher suicide rates and other hospital admissions\citep{RN1314,RN1316,RN1317,RN1318}. Heat exposures in working environments could also reduce people's capacity to deal with physical and mental task, and increase the risk of accidents, because of the excessive core body temperature and dehydration\citep{RN1319,RN1320,RN1321,RN1322}. The loss of work capacity would in turn result in loss of income, while the low income could also cause mental health problems\citep{Katz1997}.%% more physiological terms 

As for the indirect pathway, first, because of the reciprocal causation relationship between physical health problems and mental health problems\citep{RN1323,RN1324} heat exposures associated with climate change along with other climate events and indirect health risks threatening physical health will directly influence mental health \citep{RN1325,RN1326}. Second,disordered temperature associated with climate change may destroy the economic in agricultural-production-dependent communities. For example, extreme heat reduces the work capacity of laborers in farm fields \citep{RN1320}, which further destroys agricultural-supported industries in local area\citep{RN1327}. The following economic pressures would undermine social capital and then lead to mental health problems.

% Heatwaves are known to worsen mental health



https://doi.org/10.1016/j.puhe.2018.06.008 (review study - will have more to cite)
https://www.nature.com/articles/s41558-018-0222-x
https://ehp.niehs.nih.gov/doi/full/10.1289/ehp.11339
https://ijmhs.biomedcentral.com/articles/10.1186/s13033-018-0210-6
https://www.nature.com/articles/s41558-018-0102-4/
https://doi.org/10.1016/j.jhealeco.2019.102240

[ADD SUICIDE DATA TO HEALTH SECTION?]
https://www.psychologytoday.com/us/blog/greening-the-media/201809/global-warming-and-suicide

\subsection{Mental Health and Vulnerability}
% This is good, but probably should be shortened.  Maybe one para showing that income matters, one para discussing pathways, also maybe focus/change phrasing to be more on "vulnerability" -Matt

% AND mental health is strongly associated with income
% 1)Childhood 2)adulthood 3)gender
% Perceived income level

% Talk more about race
% This is interesting: https://www.pnas.org/content/118/17/e2019624118.abstract?etoc

The relationship between income, socioeconomic status (SES), well-being and mental health is one of the strongest established patterns in psycho-social literature \citep{Easterlin1974Jan, holzer1986increased, Perry1996Sep} (I cant find Ng et al. 2014). Much of the research into these patterns focuses on the inequalities and distribution of the effects on individual well-being. Subsequent work on mental health outcomes, one measure of well-being, have established strong associations between SES, income and mental health with the most and least privileged most commonly associated with the best and worse mental health experiences and outcomes, respectively \citep{Sevenson2008Aug}(cites: Ng et al. 2014). 

For example, results from a cross-sectional comparative study of socioeconomic factors and use of mental health services by people living in Ontario, Canada and the United States, indicate clear disparities in use of mental health services between those with the highest income levels and lowest mental health morbidity and those with the lowest income levels and highest mental health morbidity - those with higher incomes tended to have lower mental health morbidity issues while utilizing mental health services more than those with lower incomes and higher mental health issues \citep{Katz1997}. 

Results from a systematic review of the literature on associations between social inequalities and mental health disorders indicate a clear and prevailing link between  one or more indicators of less social privilege and higher prevalence of mental health disorders \citep{Fryers2003}, with low income one of the most consistent markers of and associations with increases in common mental health disorders. Collectively, results from the income and mental health literature indicate that socially disadvantaged populations experience significantly more frequent common mental health disorders. Scholars use mental health disparities to indicate the disproportionate amount of mental health disorders among persons of low SES \citep{RN1292}

The established association between poverty and mental health revealed that that income distribution may have a significant influence upon mental health over and above the effect of poverty \citep{HANANDITA201459}.Poverty has been identified as one of the major risk factors in mental health. For instance, some systematic reviews have shown that socioeconomically disadvantaged children and adolescents were 2-3 times more likely to have mental health problems than others\citep{REISS201324}, while the household increases in financial resources are generally associated with an overall reduction of mental health problems\citep{2015Does}. Others have argued that increases in SES can serve as a buffer against the negative impacts of difficult life experiences on mental health, particularly for individuals already under substantive mental stress \citep{Kawachi2001Sep}. Thus, SES and income are associated with mental health improvements in SES and income can serve to decrease mental health problems.

 
Several reasons could explain why poverty may effect mental health. 1) the “social causation hypothesis” which suggests that stress or deprivation or decreasing the likelihood of people getting treatment may lead to poor mental health\citep{mills2015}, while social capital could reduce the likelihood of living in poverty and reduces the risk of mental disorders such as depressive symptoms and suicidal tendencies\citep{RN1291}. 2) negative life events in a person’s life such as job loss are also associated with the risks of poverty\citep{RN1293} and mental health\citep{TAMPUBOLON201420}, and the causal inferences on the effect of poverty and mental health had also been made in several studies. For example, Tampubolon and Hanandita\citep{TAMPUBOLON201420} find that individual social capital is positively associated with mental health while adverse events were negatively associated; Chang.et.al \citep{RN1291} find that subjective and objective poverty is significantly associated with a higher risk of adverse life events, less social support and mental distress. Negative life events and social support in serial mediate the relationship between subjective poverty and mental health.

% jieliu.cnah: Could delete
For children, poverty and poor maternal mental health often co-exist and are two of the risk factors for child development\citep{LUND20111502}, and maternal mental health is significantly associated with child general psychopathology\citep{ RN1289}. Researchers also find that transition into poverty increased children’s socioemotional behavior problems and maternal psychological distress \citep{WICKHAM2017e141}.

In general, the relationship between mental health and income level is well established. In the study, we explore whether income level would influence the link between weather and the sentiment in Twitter-posed text.

https://jamanetwork.com/journals/jamapsychiatry/fullarticle/211213
https://doi.org/10.1002/9781118410868.wbehibs570
https://www.sciencedirect.com/science/article/abs/pii/S027795362030527X
https://www.sciencedirect.com/science/article/pii/S2468266717300117
https://doi.org/10.1016/S2468-2667(17)30011-7
https://link.springer.com/article/10.1007/s00127-017-1370-4
 % Jeremiah Osborne-Gowey: Not as relevant. Saved the citation but didn't find enough relevant text to include reference to here in our manuscript. Maybe this: https://journals.plos.org/plosone/article?id=10.1371/journal.pone.0116820
https://www.sciencedirect.com/science/article/abs/pii/S0143622814002537
also this: https://www.sciencedirect.com/science/article/abs/pii/S0140673614614604

\subsection{Income and Temperature}
% AND wealthier better able to cope with heatwaves 
https://www.nature.com/articles/nclimate3253
https://www.sciencedirect.com/science/article/abs/pii/S0378778818321327
https://www.mdpi.com/2225-1154/8/1/12/htm

NOT CLEAR: https://doi.org/10.1016/j.jhealeco.2019.102240 and https://www.nature.com/articles/s41558-018-0222-x find little evidence of adaptation

% BUT we know little about the interaction between heatwaves, mental health and income


% THEREFORE: we did this really novel study.
https://www.nature.com/articles/s41598-017-12961-9

%OTHER LEFTOVER TEXT TO MAYBE BRING IN LATER



Meteorological conditions can impact human physical (cite) and emotional states (cite). Emotional states and well-being are associated with physiological functioning and mental acuity which can affect social relationships, workplace productivity (cite) and health risks (cite). People that are more reliant on 1) livelihoods which require them to be outdoors (e.g., farming, agriculture, logging, etc.), 2) living and working in places where they are exposed to environmental minima and maxima, or 3) with prolonged exposure to ambient meteorological extremes are particularly vulnerable to changes in environmental exposure \citep{frimpong_heat_2017} and associated health risks (cite). Exposure to extreme environmental conditions can also have implications for workplace productivity and livelihoods \cite{kjellstrom_impact_2016}. For example, Nigerian maize farmers experience significant declines in productivity (2-8percent per degree above 17degC) along with substantive increases in health impacts including dehydration, muscle cramps, headaches and dizziness, heat exhaustion, sun stroke, and even death \cite{sadiq_impact_2019}. Other research indicates farm health and labor productivity were compromised under extreme heat and cold events in the Nepali Food Bowl region \cite{budhathoki_socio-economic_2019}. Similarly, Oregon farmers reported health impacts of working in both hot and cold conditions but the negative impacts were attenuated relative to the high heat situations \cite{bethel_heat-related_2014}. Other research found occupational injuries in Thailand increased with increasing heat exposure and occupational heat stress \cite{tawatsupa_association_2013}. Heat exposure risks are also a key factor in urban areas with particular attention needed for how the impacts of changing climate play out in health inequality \cite{friel_urban_2011}. 

Current estimates place annual workplace productivity losses due to heat exposure at 15-20percent with that rate potentially doubling by 2050 \cite{kjellstrom_heat_2016}. Projected increases in future ambient temperature suggest serious short- and long-term potential health consequences from exposure to heat stress (cite) with some estimates putting potential productivity losses at several percentage points by 2030 \cite{kjellstrom_heat_2016}, with middle- and low-income countries particularly vulnerable as they are often more reliant on physical work for their livelihoods. 

The level of changes in weather influences people’s mental health and patterns of emotions expressed. A study by Sun et al. 2018 \href{file:///Users/portia/Downloads/ijerph-16-00086-v2 20(1).pdf.}{(Portia has this PDF?)} demonstrates varied relationships between haze and negative emotions of the public under different seasons of the year. Being sad or happy influences one's style and value of reasoning. Time of day, season, location, and climate allow aggregate prediction of sentiments \cite{hannak_tweetin_2012} \href{https://www.ccs.neu.edu/~amislove/publications/Weather-ICWSM.pdf}{(link to PDF)} while positive affect from an evaluative statement enhances potential responses \cite{clore_how_2007} \href{https://www.ncbi.nlm.nih.gov/pmc/articles/PMC2483304/pdf/nihms40349.pdf}{(link to PDF)}. For instance, changes in temperature affects people’s response about weather. Depending on a reference temperature for an  area and time of year, people are more likely to comment on unusual weather for a particular place and time than on the same weather considered typical in another place \cite{moore_rapidly_2019}.

% Jeremiah Osborne-Gowey: Remove parts of this paragraph that relate to call for international research on these linkages as we now are only focusing on USA for Tweets.
Although climate change is an international phenomenon experienced in every country, its discussions vary from country to country. According to Vu and others \cite{vu_nationalizing_2019}, a country’s climate severity, economic status and governance determine variations in its media discussions. 
% Jeremiah Osborne-Gowey: I think we can remove this entire paragraph now as it's outside the focus of this paper. Might be relevant for our 2nd paper!
When removing the paragraph, cut from here and copy to the end of the doc where we have extra text saved. [DO NOT DELETE, just move from here]

% Jeremiah Osborne-Gowey: This is perhaps the only piece of this paragraph still relevant for this paper...
For example, Park and others \cite{park_mood_2013} \href{https://pdfs.semanticscholar.org/b282/feb759e57530b115dcc4bb080f96598a5246.pdf}{(link)} demonstrated that many people living in a state with higher mean temperature express more positive emotions on Twitter than those living in colder states in the US. Schmidt and others \cite{schmidt_media_2013} posit that climate severity and environmental factors related to carbon dependency influences the amount of coverage climate change receives in the media in countries like USA, Australia and Germany. Studies comparing how different countries portray climate change have been widely conducted in USA, UK, France and the Netherlands \cite{vu_nationalizing_2019}. Such comparisons are mainly between developed countries with developing countries outside the focus.  Schafer and O’Neil \cite{schafer_what_2013} advance the need for academic scholars to investigate transnational contexts within climate communication. Wealthier countries with comparatively higher GDP are likely to frame climate change as a political issue as financial resources exist for exploring research on climate change. Conversely, coverage of climate change news from poorer developing countries focus mainly on international relations. Thus, the effects of national macro economic variables that affects a country’s sociopolitical and economic development such as GDP reflects a country’s governance system in media coverage \cite{vu_what_2018}. Cultural theorists assert that, understanding such differential variations depicts how different entities interpret danger and respond to risk \cite{tansey_cultural_1999}. Inclusion of developing countries in scientific research provides an avenue for international support to such countries in responding to the effects of climate change.Yet studies attempting to explain cross-national variation in climate change public opinion is limited \cite{knight_public_2016}.

Public debates on the current consequences of climate change are discussed in both scientific and non-scientific mediums. 
% Jeremiah Osborne-Gowey: Need a better transition here between meteorological conditions and effect on mood (expressed as sentiment). Feels like we're missing a discussion (new paragraph?) just above here that discusses this. This might be where we plug in the text PAW is working on?
People can express their emotional states and feelings - sentiment - through physical, vocal and written expressions on mediums including newspapers, scientific articles, blogs and other online social media platforms. Twitter is one of the more popular social media platforms where people express their sentiments about any number of topics. Public posts on Twitter - called tweets - are widely available for public consumption and research, and offer a glimpse into collective social state. For example, text posted to these platforms can be used to gauge the public mood, assess opinions, measure brand affinity and for emergency planning and disaster response (cites). 
% Jeremiah Osborne-Gowey: Good to cite the top-cited public opinion papers, the Twitter flood study from CU, others from data science field including another Baylis paper
Sentiment is frequently employed as a correlate of emotional state or mood (cites). Weather events focus public attention in different ways of expression (cites). These data offer a unique opportunity to analyze how people express and respond to events in people's daily lives, including their exposure to ambient environmental conditions.



Here, we report on associations between meteorological conditions and expressed sentiment of public posts on Twitter from the United States of America (USA) between 2009 and 2019. This work builds on research from Baylis and others \cite{baylis_weather_2018} on weather impacts on expressed sentiment. In particular, we extend this work by including the temperature effects on sentiment in multiple regions across the USA while drawing in various climatic variables and socio-demographic data. In this research, we explore two primary questions. First, to what extent are ambient environmental conditions correlated with changes in expressed sentiment? Second, is this effect moderated by wealth?

Building on previous findings from Baylis and others, we hypothesize that local ambient environmental conditions have a strong connection how people are feeling as expressed in the sentiment of text messages posted on Twitter [H1]. Second, we hypothesize that the weather-sentiment link will be stronger in low-income areas. That is to say, people in lower income areas will be have lower sentiment scores than those in wealthier areas. [H2].
